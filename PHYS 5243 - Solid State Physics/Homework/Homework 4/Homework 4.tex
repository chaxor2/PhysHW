\documentclass{article}

\usepackage{fancyhdr}
\usepackage{extramarks}
\usepackage{amsmath}
\usepackage{amssymb}
\usepackage{graphicx}
\usepackage{pgfplotstable}

%
% Basic Document Settings
%

\topmargin=-0.45in
\evensidemargin=0in
\oddsidemargin=0in
\textwidth=6.5in
\textheight=9.0in
\headsep=0.25in

\linespread{1.1}

\pagestyle{fancy}
\lhead{\hmwkAuthorName}
\chead{\hmwkClass\ : \hmwkTitle}
\rhead{\firstxmark}
\lfoot{\lastxmark}
\cfoot{\thepage}

\renewcommand\headrulewidth{0.4pt}
\renewcommand\footrulewidth{0.4pt}

\setlength\parindent{0pt}

%
% Create Problem Sections
%

\newcommand{\enterProblemHeader}[1]{
    \nobreak\extramarks{}{Problem \arabic{#1} continued on next page\ldots}\nobreak{}
    \nobreak\extramarks{Problem \arabic{#1} (continued)}{Problem \arabic{#1} continued on next page\ldots}\nobreak{}
}

\newcommand{\exitProblemHeader}[1]{
    \nobreak\extramarks{Problem \arabic{#1} (continued)}{Problem \arabic{#1} continued on next page\ldots}\nobreak{}
    \stepcounter{#1}
    \nobreak\extramarks{Problem \arabic{#1}}{}\nobreak{}
}

\setcounter{secnumdepth}{0}
\newcounter{partCounter}
\newcounter{homeworkProblemCounter}
\setcounter{homeworkProblemCounter}{1}
\nobreak\extramarks{Problem \arabic{homeworkProblemCounter}}{}\nobreak{}

%
% Homework Problem Environment
%
% This environment takes an optional argument. When given, it will adjust the
% problem counter. This is useful for when the problems given for your
% assignment aren't sequential. See the last 3 problems of this template for an
% example.
%
\newenvironment{homeworkProblem}[1][-1]{
    \ifnum#1>0
        \setcounter{homeworkProblemCounter}{#1}
    \fi
    \section{Problem \arabic{homeworkProblemCounter}}
    \setcounter{partCounter}{1}
    \enterProblemHeader{homeworkProblemCounter}
}{
    \exitProblemHeader{homeworkProblemCounter}
}

%
% Homework Details
%   - Title
%   - Due date
%   - Class
%   - Section/Time
%   - Instructor
%   - Author
%

\newcommand{\hmwkTitle}{Homework\ \#4}
\newcommand{\hmwkDueDate}{February 16, 2015}
\newcommand{\hmwkClass}{PHYS 5243 - Solid State Physics}
\newcommand{\hmwkClassInstructor}{Professor Sheena Murphy}
\newcommand{\hmwkAuthorName}{Chase Brown}


\begin{document}
	\begin{homeworkProblem}
		Calculate the lattice constant and binding energy of NaCl ions if the ions are doubly ionized (q=2 instead of 1).

		\textbf{Solution}\\
			We know from Charles Kitell's Introduction to Solid State physics equation 3.18 that:
			\begin{equation*}
				U_{tot} = N(z\lambda e^{-\frac{R}{\rho}}-\frac{\alpha q^2}{R})
			\end{equation*}
		Now to solve for the equilibrium position $R_o$, we find the minimum of the energy with respect to $R$, $\frac{dU}{dR}=0$:
			\begin{equation*}
				R_o^2 e^{-\frac{R_o}{\rho}} = \frac{\rho \alpha q^2}{z \lambda}
			\end{equation*}
		 Using the values $z \lambda =  1.05 \times 10^{-8} ergs = 1.05 \times 10^{-15}$ J \AA \ and $\rho = 0.321$\AA \ and $\alpha = 1.747565$, and substituting $q^2$ for $\frac{q^2}{4 \pi \epsilon_o}$ we find that:
			\begin{equation*}
				\frac{q^2}{4 \pi \epsilon_o}\Big|_{q=1} = 2.307077 \times 10^{-18} J \AA
			\end{equation*}
			\begin{equation*}
				\frac{q^2}{4 \pi \epsilon_o}\Big|_{q=2} = 9.228309 \times 10^{-18} J \AA
			\end{equation*}
		Therefore we obtain:
			\begin{equation*}
				R_o^2 e^{-\frac{R_o}{0.321 \AA}}\Big|_{q=1} = \frac{0.321 \AA (1.747565)(2.307077 \times 10^{-18} J \AA)}{1.05 \times 10^{-15} J \AA } = 0.00123256 \AA^2
			\end{equation*}
			\begin{equation*}
				R_o^2 e^{-\frac{R_o}{0.321 \AA}}\Big|_{q=2} = \frac{0.321 \AA (1.747565)(9.228309 \times 10^{-18} J \AA)}{1.05 \times 10^{-15} J \AA } = 0.00493026 \AA^2
			\end{equation*}

		Solving for $R_o$ yeilds:
			\begin{equation*}
				\boxed{R_o \Big|_{q=1} = 2.81463 \AA}
			\end{equation*}
			\begin{equation*}
				\boxed{R_o \Big|_{q=2} = 2.21616 \AA}
			\end{equation*}

		The binding energy per atom is given by:
			\begin{equation*}
				\frac{U_{tot}}{N} = - \frac{\alpha q^2}{R_o} ( 1 - \frac{\rho}{R_o})
			\end{equation*}
		Therefore we find:
			\begin{equation*}
				\frac{U_{tot}}{N}\Big|_{q=1} = - \frac{1.747565 (2.307077 \times 10^{-18} J \AA)}{2.81463 \AA} (1-\frac{0.321 \AA}{2.81463 \AA})
			\end{equation*}
			\begin{equation*}
				\frac{U_{tot}}{N}\Big|_{q=2} = - \frac{1.747565 (9.228309 \times 10^{-18} J \AA)}{2.21616 \AA} (1-\frac{0.321 \AA}{2.21616 \AA})
			\end{equation*}
			\begin{equation*}
				\boxed{\frac{U_{tot}}{N}\Big|_{q=1} = -7.92 eV}
			\end{equation*}
			\begin{equation*}
				\boxed{\frac{U_{tot}}{N}\Big|_{q=2} = -38.8 eV}
			\end{equation*}
			
	\end{homeworkProblem}
		
	\pagebreak


	\begin{homeworkProblem}
		Suppose you were abe to permeate space between ions in an ionic crystal with a dielectric ($\epsilon = 81$ like in water). This reduces the Coulomb interaction by $\frac{1}{\epsilon}$.  Calculate the lattice constant and binding energy of NaCl in ths situation.  Compare the binding energy per atom with the approximate thermal energy ($kT$) at room temperature.

		\textbf{Solution}\\
			Similar to before, we have the values $z \lambda =  1.05 \times 10^{-8} ergs = 1.05 \times 10^{-15}$ J \AA \ and $\rho = 0.321$\AA \ and $\alpha = 1.747565$, and substituting $q^2$ for $\frac{q^2}{4 \pi \epsilon_o}$ we find that:
			\begin{equation*}
				\frac{q^2}{4 \pi \epsilon_o \epsilon}\Big|_{q=1} = 2.848244 \times 10^{-20} J \AA
			\end{equation*}
		Therefore we obtain:
			\begin{equation*}
				R_o^2 e^{-\frac{R_o}{0.321 \AA}}\Big|_{q=1} = \frac{0.321 \AA (1.747565)(2.848244 \times 10^{-20} J \AA)}{1.05 \times 10^{-15} J \AA } = 0.000015216 \AA^2
			\end{equation*}

		Solving for $R_o$ yeilds:
			\begin{equation*}
				\boxed{R_o \Big|_{q=1} = 4.5309 \AA}
			\end{equation*}

		The binding energy per atom is given by:
			\begin{equation*}
				\frac{U_{tot}}{N} = - \frac{\alpha q^2}{R_o} ( 1 - \frac{\rho}{R_o})
			\end{equation*}


	\end{homeworkProblem}


\end{document}
