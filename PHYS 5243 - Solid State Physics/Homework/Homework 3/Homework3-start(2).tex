\documentclass{article}

\usepackage{fancyhdr}
\usepackage{extramarks}
\usepackage{amsmath}
\usepackage{amssymb}
\usepackage{graphicx}
\usepackage{pgfplotstable}

%
% Basic Document Settings
%

\topmargin=-0.45in
\evensidemargin=0in
\oddsidemargin=0in
\textwidth=6.5in
\textheight=9.0in
\headsep=0.25in

\linespread{1.1}

\pagestyle{fancy}
\lhead{\hmwkAuthorName}
\chead{\hmwkClass\ : \hmwkTitle}
\rhead{\firstxmark}
\lfoot{\lastxmark}
\cfoot{\thepage}

\renewcommand\headrulewidth{0.4pt}
\renewcommand\footrulewidth{0.4pt}

\setlength\parindent{0pt}

%
% Create Problem Sections
%

\newcommand{\enterProblemHeader}[1]{
    \nobreak\extramarks{}{Problem \arabic{#1} continued on next page\ldots}\nobreak{}
    \nobreak\extramarks{Problem \arabic{#1} (continued)}{Problem \arabic{#1} continued on next page\ldots}\nobreak{}
}

\newcommand{\exitProblemHeader}[1]{
    \nobreak\extramarks{Problem \arabic{#1} (continued)}{Problem \arabic{#1} continued on next page\ldots}\nobreak{}
    \stepcounter{#1}
    \nobreak\extramarks{Problem \arabic{#1}}{}\nobreak{}
}

\setcounter{secnumdepth}{0}
\newcounter{partCounter}
\newcounter{homeworkProblemCounter}
\setcounter{homeworkProblemCounter}{1}
\nobreak\extramarks{Problem \arabic{homeworkProblemCounter}}{}\nobreak{}

%
% Homework Problem Environment
%
% This environment takes an optional argument. When given, it will adjust the
% problem counter. This is useful for when the problems given for your
% assignment aren't sequential. See the last 3 problems of this template for an
% example.
%
\newenvironment{homeworkProblem}[1][-1]{
    \ifnum#1>0
        \setcounter{homeworkProblemCounter}{#1}
    \fi
    \section{Problem \arabic{homeworkProblemCounter}}
    \setcounter{partCounter}{1}
    \enterProblemHeader{homeworkProblemCounter}
}{
    \exitProblemHeader{homeworkProblemCounter}
}

%
% Homework Details
%   - Title
%   - Due date
%   - Class
%   - Section/Time
%   - Instructor
%   - Author
%

\newcommand{\hmwkTitle}{Homework\ \#3}
\newcommand{\hmwkDueDate}{January 4, 2015}
\newcommand{\hmwkClass}{PHYS 5243 - Solid State Physics}
\newcommand{\hmwkClassInstructor}{Professor Sheena Murphy}
\newcommand{\hmwkAuthorName}{Chase Brown}


\begin{document}
	\begin{homeworkProblem}
		Using the following equations in Simon's Solid State Basics:
		\begin{description}
			\item[\textbf{Equation 4.3:}] \hfill
			\begin{equation*}
				N=2\sum\limits_{k}^{} n_F(\beta(\epsilon(\textbf{k})-\mu))=2\frac{V}{(2\pi)^3}\int \textbf{dk}  n_F(\beta(\epsilon(\textbf{k})-\mu))
			\end{equation*}			
	
			\item[\textbf{Equation 4.6:}]\hfill
			\begin{equation*}
				N=2\frac{V}{(2\pi)^3}(\frac{4}{3}\pi k_F^3)
			\end{equation*}

			\item[\textbf{Equation 4.7:}]\hfill
			\begin{equation*}
				E_F=\frac{\hbar^2(3\pi^2n)^{\frac{2}{3}}}{2m}
			\end{equation*}

			\item[\textbf{Equation 4.7:}]\hfill
			\begin{equation*}
				g(\epsilon)d\epsilon = \frac{(2m)^{\frac{3}{2}}}{2\pi^2\hbar^3}\epsilon^{\frac{1}{2}}d\epsilon
			\end{equation*}
				
		\end{description}

	    \hfill
	    
	    \begin{description}
		\item[\textbf{a)}] Provide the 1D and 2D analogues of these equations.
		\item[\textbf{b)}] Sketch the Density of States in 1D, 2D, and 3D.
	    \end{description}


	    \textbf{Solution} \hfill
	    
	    \begin{description}
                \item[\textbf{a)}] 
               		\begin{description}
                	        \item[\textbf{Equation 4.3:}] \hfill

					N $\equiv$ The total number of electrons within the system. \\
					There are 2 possible spin states for electrons (fermions), Therefore we have a prefector of 2. \\
                        		Due to the spacing between points in k-space $(\frac{2\pi}{L})$, the sum over all \textbf{k} is replaced by an integral multiplied by the spacing: 
					\begin{description}
						\item[\textbf{1D:} ] 
							\begin{equation*}
								\sum\limits_{k}^{} \rightarrow \frac{L}{2\pi} \int \textbf{dk}
							\end{equation*}	
					
						\item[\textbf{2D:} ] 
							\begin{equation*}
								\sum\limits_{k}^{} \rightarrow \frac{L^2}{(2\pi)^2} \int \textbf{dk} = \frac{A}{(2\pi)^3} \int \textbf{dk}
							\end{equation*}

						\item[\textbf{3D:} ] 
							\begin{equation*}
								\sum\limits_{k}^{} \rightarrow \frac{L^3}{(2\pi)^3} \int \textbf{dk} = \frac{V}{(2\pi)^3} \int \textbf{dk}
							\end{equation*}

					\end{description}

					Therefore, the total number of electrons for each dimensional system is given by:

					\begin{description}
						\item[\textbf{1D:} ] 
							\begin{equation*}
								\boxed{N_{1D} = 2\frac{L}{2\pi}\int \textbf{dk}  n_F(\beta(\epsilon(\textbf{k})-\mu))}
							\end{equation*}	
					
						\item[\textbf{2D:} ] 
							\begin{equation*}
								\boxed{N_{2D} = 2\frac{A}{(2\pi)^2}\int \textbf{dk}  n_F(\beta(\epsilon(\textbf{k})-\mu))}
							\end{equation*}

					\end{description}

                       		\item[\textbf{Equation 4.6:}]\hfill

					Solving the integral provides the solution of a sphere (for 3D) of radius $k_F$. \\
					Likewise, a circle (2D) and a line (1D) can be used with a radius or length of $k_F$.
					\begin{description}
						\item[\textbf{1D:} ] 
							
							\begin{equation*}
								N=2\frac{L}{2\pi}(k_F) = \frac{L}{\pi} k_F \\
								\Rightarrow \boxed{N_{1D}=\frac{L}{\pi} k_F}
							\end{equation*}	
					
						\item[\textbf{2D:} ] 
							\begin{equation*}
								N=2\frac{A}{(2\pi)^2}(\pi k_F^2) = \frac{A}{2\pi}k_F^2 \\
								\Rightarrow \boxed{N_{2D}=\frac{A}{2\pi}k_F^2}
							\end{equation*}

					\end{description}


                       		\item[\textbf{Equation 4.7:}]\hfill

					The Fermi energy $(E_F)$ is then found by substituting the electron density $n=\frac{N}{V}$, solving for $k_F$ and using Equation 4.4 ($E_F = \frac{\hbar^2 k_F^2}{2m}$):
					\begin{description}
						\item[\textbf{1D:} ] 
							
							\begin{equation*}
								N/L=n=\frac{k_{F,1D}}{\pi} \Rightarrow k_{F,1D} = n\pi \\
								\therefore \boxed{E_{F,1D} = \frac{\hbar^2 n^2 \pi^2}{2m}}
							\end{equation*}	
					
						\item[\textbf{2D:} ] 
							\begin{equation*}
								N/A=n=\frac{\frac{2}{(2\pi)^2}\pi k_{F,2D}^2}{\pi} \Rightarrow k_{F,2D}^2 = \frac{n(2\pi)^2}{2\pi} = 2\pi n \\
								\Rightarrow k_{F,2D} = \sqrt{2n\pi}
								\therefore \boxed{E_{F,2D} = \frac{\hbar^2 n \pi}{m}}
							\end{equation*}

					\end{description}


                       		\item[\textbf{Equation 4.10:}]\hfill

					
					\begin{description}
						\item[\textbf{1D:} ] 
							
							\begin{equation*}
								g_{1D}(\epsilon)d\epsilon=\frac{2}{2\pi}(1) dk \\
								\Rightarrow g_{1D}(\epsilon)d\epsilon = \frac{dk}{\pi} \\
								\because k = \sqrt{\frac{2\epsilon m}{\hbar^2}} \\
								\Rightarrow dk = \sqrt{\frac{m}{2\epsilon\hbar^2}} \\
								\because E_{F,1D} = \frac{\hbar^2 n^2 \pi^2}{2m} \\
								\Rightarrow \sqrt{\frac{m}{2\epsilon\hbar^2} = \sqrt{\frac{1}{4 E_{F,1D} \epsilon}} n \pi \\
								\therefore \boxed{g_{1D}(\epsilon)d\epsilon = n\sqrt{\frac{1}{4 E_{F,1D} \epsilon}}}}
							\end{equation*}	
					
						\item[\textbf{2D:} ] 


					\end{description}


                	\end{description}

			
		\item[\textbf{b)}] Sketch the Density of States in 1D, 2D, and 3D.
            \end{description}
	    

	\end{homeworkProblem}

\end{document}
