\documentclass[english, 11pt]{article}

\usepackage[T1]{fontenc}
\usepackage[utf8]{inputenc}
\usepackage{geometry}
\usepackage{url}
\usepackage{amsthm}
\usepackage{mathrsfs}
\usepackage{makeidx}
\usepackage{graphicx}
\usepackage{babel}
\usepackage{fancyhdr}
\usepackage[T1]{fontenc}
\usepackage{amsmath,amssymb}
\usepackage{url}
\usepackage{enumerate}
\usepackage{tikz}

\tikzstyle{na} = [shape=rectangle, inner sep =0pt]

\newcommand{\tocandfigures}{
	\rule[0.5ex]{1\columnwidth}{1pt}
	Last Revision: \today
	\noindent \begin{flushleft}
	\makeatletter
	\makeatother
	\setcounter{page}{1}
	\setcounter{secnumdepth}{0} %Hide section number in header
	\end{flushleft}
	\noindent \begin{flushleft}
	\tableofcontents{}
	\par\end{flushleft}
	\noindent \begin{flushleft}
	\rule[0.5ex]{1\columnwidth}{1pt}
	\par\end{flushleft}
	\noindent \begin{flushleft}
	\newpage{}
	\par\end{flushleft}
}


\newcommand{\thiscoursecode}{PHYS 5243}
\newcommand{\thiscoursename}{Solid State Physics}
\newcommand{\thisprof}{Sheena Murphy}
\newcommand{\me}{Chase Brown}
\newcommand{\thisterm}{Spring 2015}


% helpful stuff
\newcommand{\vect}[1]{\textrm{\textbf{#1}}}

% Headers
\pagestyle{fancy}

%%%%% TITLE %%%%%
\newcommand{\notefront} {
	\pagenumbering{roman}
	\begin{center}
		\textbf{\Huge{\thiscoursecode}}{\Huge \par}
		{\Large{\thiscoursename}}\\ \vspace{0.1in}
		\thisprof \ $\bullet$ \ \thisterm \ $\bullet$ \ University of Oklahoma \\
	  \end{center}
  }

\begin{document}
\tikzstyle{every picture}==[remember picture]
	\thispagestyle{empty}
	\notefront
	% Table of Contents and List of Figures
	\tocandfigures		

	
	\section{2015-01-09: Chapter 1 - About Condensed Matter Physics}
		\subsection{Syllabus}
		\underline{Read Chapters 1 and 2 before next lecture} \\
		Graduate Student $\rightarrow 15\%$ of the grade is HW. \\
		2 Midterms: Wednesday nights ($\sim$ 4 hours are given to do them). \\
		The Final counts for $\sim 25\%$ of grade for Graduate and Undergraduate Students. \\
		\mbox{Get the other books required for class $\rightarrow$ they are important!} \\
		Graduate Studnet difference $\rightarrow$ potentially a physics simulation will be required. \\

 		\begin{center}
			\line(1,0){425}
		\end{center} 
		\subsection{Class Notes}

		\begin{center} 
			\framebox{
				\tikz\node[na](word1){
					\thiscoursename
				};
			} \\
			(Condensed Matter Physics)
		\end{center} 

		
		\tikz\node[na](word2){Physics}; of complexity $\rightarrow$ statistical physics (thermodynamics)

		\begin{tikzpicture}[overlay]
			\path[->, black, thick](word1) edge [out=180, in=180] (word2);
		\end{tikzpicture}

		\begin{description}
			\item Collections of atoms
				\begin{description} 
					\item{Somewhat under atomic physics field}
					\item{Solids, liquids, and polymers}
				\end{description}
		\end{description}
		
		Hamiltonian:
		\begin{equation*}
				\hat{H} = \underbrace{\frac{\mathbf{p_n}^2}{2 M_n}}_{\substack{momentum\\of\\ions}} + 
							\underbrace{\frac{\mathbf{p_e}^2}{2 M_e}}_{\substack{momentum\\of\\electrons}} +
							\underbrace{\frac{e^2}{r_{i1}-r_{j1}}}_{\substack{repulsion\\between\\ions}}  + 
							\underbrace{\frac{e^2}{r_{i2}-r_{j2}}}_{\substack{repulsion\\between\\electrons}}  -
							\underbrace{\frac{e^2}{r_{i1}-r_{j2}}}_{\substack{attraction\\between\\electrons\ and\ ions}}
		\end{equation*}
		
		At the moment only $\sim$100 atoms can be solved (using supercomputer) $\rightarrow$ very difficult! \\
		\begin{description}
			\item Emergent phenomenon is common
			\begin{description}
				\item Superconductivity is emergent from collection of atoms
			\end{description}
		\end{description}
	
		\pagebreak


	\pagebreak
	


	\section{2015-02-20: Chapter 1 (Kittel) - Crystal Structure}
		Test on Everything but Crystal Structure.
		Closed book but will provide equations.
		\subsection{Primitive Cells}
			Crystal Structure handout.
			\begin{description}
				\item[(100)]  \ 
					plane of atoms.
				\item[$\{$100$\}$] \
					family of planes.
				\item[\ [100] ] \
					direction.
			\end{description}

\end{document}

