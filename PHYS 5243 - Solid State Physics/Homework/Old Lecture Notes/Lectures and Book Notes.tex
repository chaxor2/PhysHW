\documentclass[english, 11pt]{article}

\usepackage[T1]{fontenc}
\usepackage[utf8]{inputenc}
\usepackage{geometry}
\usepackage{url}
\usepackage{amsthm}
\usepackage{mathrsfs}
\usepackage{makeidx}
\usepackage{graphicx}
\usepackage{babel}
\usepackage{fancyhdr}
\usepackage[T1]{fontenc}
\usepackage{amsmath,amssymb}
\usepackage{url}
\usepackage{enumerate}
\usepackage{tikz}

\tikzstyle{na} = [shape=rectangle, inner sep =0pt]

\newcommand{\tocandfigures}{
	\rule[0.5ex]{1\columnwidth}{1pt}
	Last Revision: \today
	\noindent \begin{flushleft}
	\makeatletter
	\makeatother
	\setcounter{page}{1}
	\setcounter{secnumdepth}{0} %Hide section number in header
	\end{flushleft}
	\noindent \begin{flushleft}
	\tableofcontents{}
	\par\end{flushleft}
	\noindent \begin{flushleft}
	\rule[0.5ex]{1\columnwidth}{1pt}
	\par\end{flushleft}
	\noindent \begin{flushleft}
	\newpage{}
	\par\end{flushleft}
}


\newcommand{\thiscoursecode}{PHYS 5243}
\newcommand{\thiscoursename}{Solid State Physics}
\newcommand{\thisprof}{Sheena Murphy}
\newcommand{\me}{Chase Brown}
\newcommand{\thisterm}{Spring 2015}



% Headers
\pagestyle{fancy}

%%%%% TITLE %%%%%
\newcommand{\notefront} {
	\pagenumbering{roman}
	\begin{center}
		\textbf{\Huge{\thiscoursecode}}{\Huge \par}
		{\Large{\thiscoursename}}\\ \vspace{0.1in}
		\thisprof \ $\bullet$ \ \thisterm \ $\bullet$ \ University of Oklahoma \\
	  \end{center}
  }

\begin{document}
\tikzstyle{every picture}==[remember picture]
	\thispagestyle{empty}
	\notefront
	% Table of Contents and List of Figures
	\tocandfigures		

	
	\section{Chapter 1 - About Condensed Matter Physics - (2015-01-09)}
		\subsection{Syllabus}
		\underline{Read Chapters 1 and 2 before next lecture} \\
		Graduate Student $\rightarrow 15\%$ of the grade is HW. \\
		2 Midterms: Wednesday nights ($\sim$ 4 hours are given to do them). \\
		The Final counts for $\sim 25\%$ of grade for Graduate and Undergraduate Students. \\
		\mbox{Get the other books required for class $\rightarrow$ they are important!} \\
		Graduate Studnet difference $\rightarrow$ potentially a physics simulation will be required. \\

 		\begin{center}
			\line(1,0){425}
		\end{center} 
		\subsection{Class Notes}

		\begin{center} 
			\framebox{
				\tikz\node[na](word1){
					\thiscoursename
				};
			} \\
			(Condensed Matter Physics)
		\end{center} 

		
		\tikz\node[na](word2){Physics}; of complexity $\rightarrow$ statistical physics (thermodynamics)

		\begin{tikzpicture}[overlay]
			\path[->, black, thick](word1) edge [out=180, in=180] (word2);
		\end{tikzpicture}

		\begin{description}
			\item Collections of atoms
				\begin{description} 
					\item{Somewhat under atomic physics field}
					\item{Solids, liquids, and polymers}
				\end{description}
		\end{description}
		
		Hamiltonian:
		\begin{equation*}
				\hat{H} = \underbrace{\frac{\mathbf{p_n}^2}{2 M_n}}_{\substack{momentum\\of\\ions}} + 
							\underbrace{\frac{\mathbf{p_e}^2}{2 M_e}}_{\substack{momentum\\of\\electrons}} +
							\underbrace{\frac{e^2}{r_{i1}-r_{j1}}}_{\substack{repulsion\\between\\ions}}  + 
							\underbrace{\frac{e^2}{r_{i2}-r_{j2}}}_{\substack{repulsion\\between\\electrons}}  -
							\underbrace{\frac{e^2}{r_{i1}-r_{j2}}}_{\substack{attraction\\between\\electrons\ and\ ions}}
		\end{equation*}
		
		At the moment only $\sim$100 atoms can be solved (using supercomputer) $\rightarrow$ very difficult! \\
		\begin{description}
			\item Emergent phenomenon is common
			\begin{description}
				\item Superconductivity is emergent from collection of atoms
			\end{description}
		\end{description}

		\subsection{Book Notes}
			\subsubsection{1.1 - What is Condensed Matter Physics?}
				\begin{description}
					\item[Number of consituents is large]
					\item[interactions among constituents is strong]
				\end{description}
			\subsubsection{1.2 - Why study Condensed Matter Physics?}
				\begin{description}
					\item[Good Questions] \

						Why are metals shiny and cold? \\
						Why is glass transparent? \\
						Why is water fluid, why is it wet? \\
						Why is rubber soft?\\
					\item[Engineering] 
		
					\item[Awesomeness] \

						Higgs-Anderson mechanism $\rightarrow$ ties to Higgs Boson and superconductivity (Anderson coined Condensed Matter) \\
						Renormalization group \\
						Topological QFT $\rightarrow$ in lab of CMP \\
						black hole string theory $\rightarrow$ CMP
					\item[reductionism doesn't work] \

						Just accept it ..... :(
					\item[QM and Stat Mech are basis for CMP] 
				\end{description}
				
			\subsubsection{1.3 - Why Solid State?}
				Subfield of CMP $\rightarrow$ very large
	
		\pagebreak

		\section{Chapter 2 - Heat Capacity and Specific Heat} 

		$C = \frac{dE}{dT}$ \\
		How much energy you need to increase the temperature.\\
		$C_v = C_p$ for solids, so we do not need to specify $C_{v,p}$ subscripts.\\
		Heat Capacity per mole at room temperature is $3R$. (for solids) \\
		\begin{equation*}
			R = k_B N_A
		\end{equation*}
		How do we know? \\
		Start with the heat capacity per atom $\rightarrow$ which we get from the energy for each atom. \\
		
		We construct a 3D particle in a box connected by springs along each axis and find the energy:
		\begin{equation*}
			E = \underbrace{\frac{1}{2} m v_x^2 + \frac{1}{2} m v_y^2 + \frac{1}{2} m v_z^2}_{kinetic\ energy} +
				\underbrace{\frac{1}{2} k_x^2 + \frac{1}{2} k_y^2 + \frac{1}{2} k_z^2}_{potential\ energy}
		\end{equation*}

		\underline{Equipartition of energy}\\
			each DOF gives $\frac{1}{2} k_B T$  (but only when quadratic!  (power of 2))
		
		Therefore, for solids $\rightarrow 6 ]frac{1}{2} k_B T = 3 k_B T$. \\
		$\Rightarrow \langle E\rangle = 3 k_B T$. \\
		and Law of Dulon Petit (1819) is $C = \frac{d \langle E\rangle}{dT} = 3k_B$ (or $3R$ for molar).\\
		
		Temperature Dependence:
		\begin{tikzpicture}
			% horizontal axis
			\draw[->] (0,0) -- (6,0) node[anchor=north] {$T$};
			% ranges
			\draw (1,3.5) node{{$T^2$}};
			
			% vertical axis
			\draw[->] (0,0) -- (0,4) node[anchor=east] {$U_s,\varPsi_s$};
			% nominal speed
			\draw[dotted] (2,0) -- (2,4);
			
			% Us
			\draw[thick] (0,0) -- (2,2) -- (6,2);
			\draw (1,1.5) node {$U_s$}; %label
			
			% Psis
			\draw[thick,dashed] (0,3) -- (2,3) parabola[bend at end] (6,1);
			\draw (2.5,3) node {$\varPsi_s$}; %label
		\end{tikzpicture}

	\pagebreak
	


	\section{2015-02-20: Chapter 1 (Kittel) - Crystal Structure}
		Test on Everything but Crystal Structure.
		Closed book but will provide equations.
		\subsection{Primitive Cells}
			Crystal Structure handout.
			\begin{description}
				\item[(100)]  \ 
					plane of atoms.
				\item[$\{$100$\}$] \
					family of planes.
				\item[\ [100] ] \
					direction.
			\end{description}

\end{document}

