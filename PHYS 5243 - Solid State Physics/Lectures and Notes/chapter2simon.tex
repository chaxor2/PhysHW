\section{Chapter 2 - Heat Capacity and Specific Heat} 

		$C = \frac{dE}{dT}$ \\
		How much energy you need to increase the temperature.\\
		$C_v = C_p$ for solids, so we do not need to specify $C_{v,p}$ subscripts.\\
		Heat Capacity per mole at room temperature is $3R$. (for solids) \\
		\begin{equation*}
			R = k_B N_A
		\end{equation*}
		How do we know? \\
		Start with the heat capacity per atom $\rightarrow$ which we get from the energy for each atom. \\
		
		We construct a 3D particle in a box connected by springs along each axis and find the energy:
		\begin{equation*}
			E = \underbrace{\frac{1}{2} m v_x^2 + \frac{1}{2} m v_y^2 + \frac{1}{2} m v_z^2}_{kinetic\ energy} +
				\underbrace{\frac{1}{2} k_x^2 + \frac{1}{2} k_y^2 + \frac{1}{2} k_z^2}_{potential\ energy}
		\end{equation*}

		\underline{Equipartition of energy}\\
			each DOF gives $\frac{1}{2} k_B T$  (but only when quadratic!  (power of 2))
		
		Therefore, for solids $\rightarrow 6 ]frac{1}{2} k_B T = 3 k_B T$. \\
		$\Rightarrow \langle E\rangle = 3 k_B T$. \\
		and Law of Dulon Petit (1819) is $C = \frac{d \langle E\rangle}{dT} = 3k_B$ (or $3R$ for molar).\\
		
		Temperature Dependence:
		\begin{tikzpicture}
			% horizontal axis
			\draw[->] (0,0) -- (6,0) node[anchor=north] {$T$};
			% ranges
			\draw (1,3.5) node{{$T^2$}};
			
			% vertical axis
			\draw[->] (0,0) -- (0,4) node[anchor=east] {$U_s,\varPsi_s$};
			% nominal speed
			\draw[dotted] (2,0) -- (2,4);
			
			% Us
			\draw[thick] (0,0) -- (2,2) -- (6,2);
			\draw (1,1.5) node {$U_s$}; %label
			
			% Psis
			\draw[thick,dashed] (0,3) -- (2,3) parabola[bend at end] (6,1);
			\draw (2.5,3) node {$\varPsi_s$}; %label
		\end{tikzpicture}